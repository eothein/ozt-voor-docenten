\section{Vakliteratuur, literatuurstudie}

\sectionframelogo{Kan in functie van de onderzoeksvragen geschikte vakliteratuur evalueren, selecteren en verwerken in een literatuurstudie}

\begin{frame}{Doelstelling}
  
  Aan het eind van dit deel kan je als docent:
  
  \begin{itemize}
    \item Je een idee vormen van wat studenten hierover (zouden moeten) weten;
    \item Je studenten doorverwijzen naar beschikbare info;
    \item Een draft van een literatuurstudie beoordelen en feedback geven;
  \end{itemize}
\end{frame}

\begin{frame}{Cursusinhoud}
  \begin{itemize}
    \item (Oefeningen)les over \LaTeX{} en literatuurstudie
    \item Niet-periodegebonden evaluatie
      \begin{itemize}
        \item formuleren bachelorproefonderwerp
        \item mini-onderzoek
      \end{itemize}
  \end{itemize}
\end{frame}

\begin{frame}{Studiemateriaal}
  \begin{itemize}
    \item Slides les
    \item Praktische gids voor de bachelorproef: \url{https://github.com/HoGentTIN/bachproef-gids}
  \end{itemize}
\end{frame}

\begin{frame}{Voorbereiding}

  Software-installatie
  
  \begin{itemize}
    \item Mik\TeX{}, \TeX{}Studio (\LaTeX{})
    \item JabRef (reference manager)
    \item Git, Github
  \end{itemize}
\end{frame}

