\section{Vakliteratuur, literatuurstudie}

\sectionframelogo{Kan in functie van de onderzoeksvragen geschikte vakliteratuur evalueren, selecteren en verwerken in een literatuurstudie}

\subsection{Inleiding}
\begin{frame}{Doelstelling}
  
  Aan het eind van dit deel kan je als docent:
  
  \begin{itemize}
    \item Je een idee vormen van wat studenten hierover (zouden moeten) weten;
    \item Je studenten doorverwijzen naar beschikbare info;
    \item Een draft van een literatuurstudie beoordelen en feedback geven;
  \end{itemize}
\end{frame}

\begin{frame}{Cursusinhoud}
  \begin{itemize}
    \item (Oefeningen)les over \LaTeX{} en literatuurstudie
    \item Niet-periodegebonden evaluatie
      \begin{itemize}
        \item formuleren bachelorproefonderwerp
        \item mini-onderzoek
      \end{itemize}
  \end{itemize}
\end{frame}

\begin{frame}{Studiemateriaal}
  \begin{itemize}
    \item Slides les
    \item HoGent bib: \url{https://bib.hogent.be/how-to/in-de-byb/}
    \item Praktische gids voor de bachelorproef: \url{https://github.com/HoGentTIN/bachproef-gids}
  \end{itemize}
\end{frame}

\begin{frame}{Voorbereiding}

  Software-installatie
  
  \begin{itemize}
    \item Mik\TeX{}, \TeX{}Studio (\LaTeX{})
    \item JabRef (reference manager)
    \item Git, Github
  \end{itemize}
\end{frame}

\subsection{Informatie opzoeken}
\begin{frame}{Informatie opzoeken}
  \framesubtitle{Soorten bronnen}
  
  \begin{description}
    \item[Primaire] kennis die je zelf vergaart tijdens je onderzoek.\\bv. metingen uit experimenten, resultaten van enquêtes, transcripties van interviews, enz.
    \item[Secundaire] publicatie van kennis, onderzoek, enz.~door anderen.\\bv. artikels in vaktijdschriften, boek, presentatie op een conferentie, enz.
    \item[Tertiaire] zoekindexen en encyclopedieën.\\bv. Google Scholar, Web of Science, Elsevier ScienceDirect, Arxiv.org, Wikipedia, about.com, Webopedia, enz.
  \end{description}
  
  \brightbox{In een scriptie mag je enkel verwijzen naar \textcolor{HoGentAccent6}{secundaire} bronnen.}
\end{frame}

\begin{frame}{Informatie opzoeken}
  \framesubtitle{Soorten publicaties (secundaire bronnen)}
  
  \begin{itemize}
    \item Artikel in \emph{wetenschappelijk} tijdschrift (\emph{Journal})
    \item Artikel in \emph{vaktijdschrift}
    \item Presentatie op \emph{wetenschappelijke} of vakconferentie
    \item Thesis (PhD, Master, Bachelor)
    \item Boek, handleiding
    \item White paper
    \item Blogartikel
  \end{itemize}
\end{frame}

\begin{frame}{Informatie opzoeken}
  \framesubtitle{Doe de CRAP test!}
  
\begin{description}
  \item[Currency] (actualiteit) is de bron voldoende recent voor het onderwerp?
  \item[Reliability/Relevance] (betrouwbaarheid/relevantie) is de inhoud goed onderbouwd? Wordt er naar bronnen verwezen? Is de inhoud relevant voor jouw onderzoek?
  \item[Authority] (autoriteit) heeft de auteur voldoende gezag om over het onderwerp een uitspraak te doen? Gaat het over een persoon of een organisatie?
  \item[Point of view] (objectiviteit) wat is de intentie van de auteur? Wat wil die bereiken (verkopen)?
\end{description}
\end{frame}

\begin{frame}{Informatie opzoeken}
  \framesubtitle{Startpunten}
  
  \begin{itemize}
    \item Google Scholar: wetenschappelijke literatuur\\\url{http://scholar.google.com/}
    \item Wikipedia (\emph{tertiaire} bron!)
    \item HoGent bib (incl. scripties): \url{http://bib.hogent.be/}
    \item Apollox: \url{https://apollox.hogent.be/}
    \begin{itemize}
      \item Elsevier ScienceDirect
      \item Springer Online Journals
      \item Web of Science
    \end{itemize}
  \end{itemize}

\end{frame}

\subsection{Referenties bijhouden}
\begin{frame}{Referenties bijhouden}
\framesubtitle{Doel van de referentielijst}

Lezers toelaten:

\begin{itemize}
  \item De gerefereerde bronnen op te zoeken
  \item Kwaliteit bronnen zelf te beoordelen
\end{itemize}

\pause

Strikte, vastgelegde vorm:

\begin{itemize}
  \item Vastgelegde stijl. HoGent: American Psychological Association
  \item Vaste volgorde (scriptie: alfabetisch)
  \item Lijst URLs is onvoldoende!
\end{itemize}

\pause

\brightbox{Gebruik \textcolor{HoGentAccent6}{referentie-software} zoals JabRef om je referentielijst op te maken!}
\end{frame}

\begin{frame}{Referenties bijhouden}
  \framesubtitle{Tips JabRef}
  
  Voor meer details, zie de Bachelorproef-gids!
  
  \begin{itemize}
    \item Stel JabRef in voor ``Biblatex''
    \item Hou PDFs bij in aparte map
    \item Voor elk record, minstens: Auteur, \alert{jaartal}, titel
    \item Afhankelijk van de bron aanvullen met andere info
    \begin{itemize}
      \item vb. Electronic: urldate (= datum raadplegen)
    \end{itemize}
    \item Vul ook ``extra'' info in, bv. Abstract, URL, PDF, keywords, \ldots
    \item Auteursveld: 
    \begin{itemize}
      \item Familienaam, Voornaam \alert{and} Familienaam, Voornaam \alert{and} Familienaam, Voornaam
      \item Naam organisatie tussen accolades: \{The Linux Foundation\}
    \end{itemize}
  \end{itemize}
  
\end{frame}
