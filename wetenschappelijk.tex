\section{Kan de data methodologisch verantwoord verzamelen}
\sectionframelogo{}
\subsection{De Wetenschappelijke methode}

\begin{frame}
  \scaledimg{img/les1-01}
\end{frame}


\begin{frame}
  \frametitle{De wetenschappelijke methode}

  Aan de hand van \textbf{empirisch onderzoek} zijn we geïnteresseerd in volgende zaken:

  \begin{enumerate}
    \item Exploratie
    \item Beschrijving
    \item Voorspelling
    \item Controle
  \end{enumerate}
\end{frame}

\begin{frame}
  \frametitle{De wetenschappelijke methode}
    \begin{itemize}
      \item Generalisatie
        \begin{itemize}
          \item Bv. ``De LAMP stack heeft een algemeen snellere uitvoeringstijd dan een WAMP stack''
        \end{itemize}
      \item Verstaan, begrijpen
        \begin{itemize}
          \item De WAMP stack voegt een extra delay toe aan de uitvoeringstijd door het gebruik van Windows 
          \item Theorieontwikkeling
        \end{itemize}
    \end{itemize}
\end{frame}

\begin{frame}
  \frametitle{Het onderzoeksproces}

  \begin{columns}
  \column{\dimexpr\paperwidth}
  \begin{center}
    \begin{tikzpicture}[
      auto,
      thick,
      ->,
      >=stealth',
      shorten >=1pt,
      node distance=2cm,
      fase/.style={
        shape=rectangle split,
        rectangle split parts=2,
        ,
        draw}]


      \node[fase] (1) {
        1. Formuleren probleemstelling
        \nodepart{second}
        \small{Wat is de onderzoeksvraag?}
      };
      \uncover<2->{\node[fase] (2) [below of=1] {
        2. Exacte informatiebehoefte definiëren
        \nodepart{second}
        \small{Welke specifieke vragen moeten we stellen?}
      };}
      \uncover<3->{\node[fase] (3) [below of=2] {
        3. Uitvoeren onderzoek
        \nodepart{second}
        \small{Enquêtes, simulaties, \ldots}
      };}
      \uncover<4->{\node[fase] (4) [right of=1, node distance=6cm] {
        4. Verwerken gegevens
        \nodepart{second}
        \small{Statistische software}
      };}
      \uncover<5->{\node[fase] (5) [below of=4] {
        5. Analyseren gegevens
        \nodepart{second}
        \small{Uitvoeren statistische methodes}
      };}
      \uncover<6->{\node[fase] (6) [below of=5] {
        6. Conclusies schrijven
        \nodepart{second}
        \small{Schrijven onderzoeksverslag}
      };}

      \uncover<2->{\draw (1) -- (2);}
      \uncover<3->{\draw (2) -- (3);}
      \uncover<4->{\draw (3.east) to [out=0,in=180] (4.west);}
      \uncover<5->{\draw (4) -- (5);}
      \uncover<6->{\draw (5) -- (6);}
    \end{tikzpicture}
  \end{center}
\end{columns}

\end{frame}


\begin{frame}
  \frametitle{Meetniveaus}

  Kwalitatieve schalen:

  \begin{description}
    \item[Nominaal] Categorieën. Bv. Geslacht, ras, land, vorm, \ldots
    \item[Ordinaal] Volgorde. Bv. militaire rang, opleidingsniveau, \ldots
  \end{description}

\end{frame}

\begin{frame}
  \frametitle{Meetniveaus}

  Kwantitatieve schalen:

  \begin{description}
    \item[Interval] Meting: getal + meeteenheid, nulpunt niet belangrijk\\
      bv. 20°C - 15°C = 5°C, maar 20°C is \emph{NIET} 1/3 warmer dan 15°C
    \item[Ratio] Meting t.o.v. absoluut nulpunt. bv. Afstand (m), energie (J), massa (kg), \ldots\\
      bv. 20m is \emph{wel} 1/3 langer dan 15m
  \end{description}
\end{frame}
