\section{Kan methodologisch verantwoorde analyses maken van de verzamelde data }
\sectionframe{}

\subsection{Centrum \& Spreidingsmaten}
\begin{frame}{Beschrijvende statistiek - centrummaten}
	
	\begin{description}
		\item[gemiddelde] de som van alle waarden gedeeld door het aantal waarden
		\item[mediaan] sorteer de waarden en kies het middelste nummer
		\item[modus] is het vaakst voorkomende getal in een reeks getallen
	\end{description}
\end{frame}

\begin{frame}{Beschrijvende statistiek - spreidingsmaten}
	\begin{description}
		\item[bereik] van een reeks getallen is de absolute waarde van het verschil tussen het grootste en kleinste getal in de reeks
		\item[kwartielen] van een gesorteerde reeks getallen zijn de waarden die de lijst in vier gelijke delen verdeelt. Elk deel vormt dus een kwart van de dataset. Men spreekt van een eerste, tweede en derde kwartiel, genoteerd als resp,~$Q_1$, $Q_2$, $Q_3$
		\item[variantie] is het gemiddelde gekwadrateerde verschil tussen de elementen van de dataset en zijn gemiddelde
		\item[standaardafwijking] is de vierkantswortel van de variantie
	\end{description}
\end{frame}

\begin{frame}{Grafische voorstelling}
	\begin{description}
		\item[histogram] de grafische weergave van de frequentieverdeling van in klassen gegroepeerde data, afkomstig uit een continue kansverdeling 
		\item[boxplot] een grafische weergave van de vijf-getallensamenvatting.
		\item[QQPlot] Q-Q plot vergelijkt de verdeling van een gegeven variabele grafisch naar een theoretische verdeling door de quartilen t.o.v. elkaar uit te zetten 
	\end{description}
\end{frame}

\begin{frame}{Beschrijvende statstiek - R}
	De studenten leren in onderzoekstechnieken werken met R.
	
	Google definitie: 
	\begin{center}
		\textit{R is een programmeertaal en omgeving voor statistische analyse en grafische presentatie van gegevens. R is de open source versie van S-plus. Het is vrij verkrijgbare software en maakt deel uit van het GNU (Gnu's Not Unix) project van de stichting Free Software Foundation.}
	\end{center}
\end{frame}

\begin{frame}{Demo beschrijvende statistiek in R}
\begin{center}
		Handen uit de mouwen! 
\end{center}
\includegraphics[width=\textwidth]{img/handen.png}
\end{frame}

\begin{frame}{Waarom (meestal) geen taartdiagram}
	Zie artikel van \href{http://www.businessinsider.com/pie-charts-are-the-worst-2013-6?IR=T}{The Worst Chart In The World - Walter Hickey}
\end{frame}

\begin{frame}{Waarom STDEV}
	\begin{center}
		Handen uit de mouwen! 
	\end{center}
	\includegraphics[width=\textwidth]{img/handen.png}
\end{frame}

\subsection{Valkuilen van grafieken}
\sectionframelogo{Valkuilen van grafieken}

\begin{frame}
	\frametitle{Data-ambiguïteit}
	
	= Vergeten aan te duiden wat de data betekent.
	\vspace{1cm}
	
	\begin{columns}
		\column{.5\textwidth}
		\includegraphics[width=\textwidth]{img/les2-02}
		\column{.5\textwidth}
		Tips:
		\begin{itemize}
			\item Benoem je assen
			\item Geef een duidelijke titel
			\item Benoem de meeteenheid (en evt. grootorde)
			\item Voeg een bijschrift toe met uitleg over de grafiek
		\end{itemize}
	\end{columns}
\end{frame}

\begin{frame}
	\frametitle{Data distortion}
	
	= Verkeerde conclusies laten trekken uit een grafische voorstelling
	
	\begin{center}
		\includegraphics[width=.7\textwidth]{img/les2-03}
	\end{center}
\end{frame}

\begin{frame}
	\frametitle{Data distortion}
	
	\begin{center}
		\includegraphics[width=.8\textwidth]{img/les2-03-2}
	\end{center}
\end{frame}

\begin{frame}
	\frametitle{Data distraction}
	
	\begin{itemize}
		\item Vermijd toeters en bellen in je grafieken
		\item Minimaliseer inkt tot data ratio
	\end{itemize}
	
	\centering
	\includegraphics[width=.4\textwidth]{img/les2-04}
	\includegraphics[width=.4\textwidth]{img/les2-05}
	
	\includegraphics[width=.4\textwidth]{img/les2-06}
	\includegraphics[width=.4\textwidth]{img/les2-07}
\end{frame}

\begin{frame}
	\frametitle{Anscombe's Quartet}
	
	\centering
	\includegraphics[width=.8\textwidth]{img/anscombes_quartet}
	
	Vier verschillende datasets met dezelfde statistische eigenschappen. Deze tonen het belang aan van data-visualisatie.
\end{frame}





