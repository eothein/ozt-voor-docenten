\section{Kan methodologisch verantwoorde analyses maken van de verzamelde data }
\sectionframe{}

\begin{frame}{Beschrijvende statistiek - centrummaten}
	
	\begin{description}
		\item[gemiddelde] de som van alle waarden gedeeld door het aantal waarden
		\item[mediaan] sorteer de waarden en kies het middelste nummer
		\item[modus] is het vaakst voorkomende getal in een reeks getallen
	\end{description}
\end{frame}

\begin{frame}{Beschrijvende statistiek - spreidingsmaten}
	\begin{description}
		\item[bereik] van een reeks getallen is de absolute waarde van het verschil tussen het grootste en kleinste getal in de reeks
		\item[kwartielen] van een gesorteerde reeks getallen zijn de waarden die de lijst in vier gelijke delen verdeelt. Elk deel vormt dus een kwart van de dataset. Men spreekt van een eerste, tweede en derde kwartiel, genoteerd als resp,~$Q_1$, $Q_2$, $Q_3$
		\item[variantie] is het gemiddelde gekwadrateerde verschil tussen de elementen van de dataset en zijn gemiddelde
		\item[standaardafwijking] is de vierkantswortel van de variantie
	\end{description}
\end{frame}